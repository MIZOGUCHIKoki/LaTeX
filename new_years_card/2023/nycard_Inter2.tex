\documentclass[twocolumn,fontsize=9pt]{jlreq}
\usepackage[papersize={148truemm, 100truemm}, top=15truemm,bottom=15truemm,left=10truemm,right=10truemm,headheight=20pt]{geometry}
\usepackage{fancyhdr}
\usepackage{tcolorbox,fancybox,framed}
\usepackage{multirow}
\usepackage{amsmath}
\tcbuselibrary{skins,raster}
\fancypagestyle{fancy}{
\fancyhf{}
\renewcommand{\headrulewidth}{0pt}
\renewcommand{\footrulewidth}{0pt}
\fancyhead[R]{2023年}
\fancyfoot[R]{Created by Lua-\LaTeX}
}
\fancyput(-10truemm,-50truemm){
	\color{black!10}{
		\rotatebox{50}{
			\scalebox{10}{卯}     
		}
	}
}
\pagestyle{fancy}
\begin{document}
\twocolumn[
    \begin{center}
        {\LARGE 謹賀新年}
    \end{center}
    \vspace{0.5em}
]
{\small
    謹んで新年のご挨拶を申し上げます.\par
    旧年中は大変お世話になりありがとうございました.\par
    本年も変わらぬご厚誼のほどお願い申し上げます.
    \begin{flushright}
        令和5年元旦
    \end{flushright}
}
\hrulefill\\
\noindent\textbf{数学小ネタ}\ \ {\small フィボナッチ数列の一般項}\\
\(F_{n+2}=F_{n+1}+F_{n},F_0=0,F_1=1\)
\begin{equation}
    \begin{aligned}
        F_n  & = \frac{(\phi)^n – \left( -\frac{1}{\phi} \right)^n }{\sqrt{5}} \\
        \phi & = \frac{1+\sqrt{5}}{2}
    \end{aligned}
\end{equation}
\newpage
\begin{tcolorbox}[
        enhanced,
        title={\bfseries 差出人情報},
        boxed title style={skin=enhancedfirst jigsaw,bottom=0mm,arc=1mm,boxrule=0mm},
        sharp corners=northwest,
        arc=1mm,
        attach boxed title to top left={yshift=-.5mm},
        colframe=black!80,
        colbacktitle=black!80,
        left=1mm,right=1mm
    ]
    \begin{center}
        溝口\ 洸熙
    \end{center}
    \tcblower
    {{\small\ttfamily koki.mizoguchi@aol.com}}\\
    {{\small\ttfamily https://github.com/MIZOGUCHIKoki}}
\end{tcolorbox}
\begin{center}
    \includegraphics[scale=0.23]{fig.png}{\Huge\(=7\times 17^2\)}
\end{center}
\end{document}