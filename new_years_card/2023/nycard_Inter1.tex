\documentclass[twocolumn,fontsize=9pt]{jlreq}
\usepackage[papersize={148truemm, 100truemm}, top=15truemm,bottom=15truemm,left=10truemm,right=10truemm,headheight=20pt]{geometry}
\usepackage{fancyhdr}
\usepackage{tcolorbox,fancybox,framed}
\usepackage{multirow}
\usepackage{amsmath}
\tcbuselibrary{skins,raster}
\fancypagestyle{fancy}{
\fancyhf{}
\renewcommand{\headrulewidth}{0pt}
\renewcommand{\footrulewidth}{0pt}
\fancyhead[R]{2023年}
\fancyfoot[R]{Created by Lua-\LaTeX}
}
\fancyput(-10truemm,-50truemm){
	\color{black!10}{
		\rotatebox{50}{
			\scalebox{10}{卯}     
		}
	}
}
\pagestyle{fancy}
\begin{document}
\twocolumn[
    \begin{center}
        {\LARGE 謹賀新年}
    \end{center}
    \vspace{0.5em}
]
{\small
    謹んで新年のご挨拶を申し上げます.\par
    旧年中は大変お世話になりありがとうございました.\par
    本年も変わらぬご厚誼のほどお願い申し上げます.
    \begin{flushright}
        令和5年元旦
    \end{flushright}
}
\noindent{\bfseries 数学小ネタ}\dotfill\\
{\scriptsize {\ttfamily 2023}のIEEE754における単精度浮動小数点数表現}
\begin{equation}
    \begin{aligned}
        2023_{(10)} & =11111100111_{(2)}         \\
                    & =1.1111100111\times 2^{10}
    \end{aligned}
\end{equation}
{\scriptsize {\bfseries 符号部}{\ttfamily 0},{\bfseries 指数部}{\ttfamily 137},{\bfseries 仮数部}{\ttfamily 小数点以下23Bit}}\\
従って,\fbox{\ttfamily 0 10001001 111110011100...0}
\newpage
\begin{tcolorbox}[
        enhanced,
        title={\bfseries 差出人情報},
        boxed title style={skin=enhancedfirst jigsaw,bottom=0mm,arc=1mm,boxrule=0mm},
        sharp corners=northwest,
        arc=1mm,
        attach boxed title to top left={yshift=-.5mm},
        colframe=black!80,
        colbacktitle=black!80,
        left=1mm,right=1mm
    ]
    \begin{center}
        溝口\ 洸熙
    \end{center}
    \tcblower
    {{\small\ttfamily koki.mizoguchi@aol.com}}\\
    {{\small\ttfamily https://github.com/MIZOGUCHIKoki}}
\end{tcolorbox}
\includegraphics[scale=0.24]{fig.png}{\LARGE\(=7\times 17^2\)}
\end{document}