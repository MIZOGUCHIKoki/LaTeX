\documentclass[twocolumn,fontsize=9pt]{jlreq}
\usepackage[papersize={148truemm, 100truemm}, top=5truemm,bottom=5truemm,left=5truemm,right=5truemm]{geometry}
\usepackage{fancybox,ascmac}
\usepackage{xcolor,graphicx,url}
\usepackage{listings,framed}
\usepackage{amsmath}
\lstset{
    language=C,
    breaklines=true,
    breakindent=10pt,
    basicstyle=\ttfamily\scriptsize,
    frame=TB,
    keywordstyle={\color{red!70!black}\bfseries\ttfamily},
    commentstyle={\color{green!50!black}\ttfamily},
    stringstyle={\color{blue}\ttfamily},
    morekeywords={pipe, fork, pid_t, printf, sprintf, write, read},
    numbers=left,
    numberstyle={\ttfamily\scriptsize},
    numbersep = {1pt}
}

\fancyput(-20truemm,-60truemm){
	\color{green!10}{
		\rotatebox{50}{
			\scalebox{10}{巳年}     
		}
	}
}
\begin{document}
\twocolumn[
    \begin{center}
        {\LARGE 謹賀新年}
    \end{center}
    \vspace{0.5em}
]
{\small
    謹んで新年のご挨拶を申し上げます\par
    旧年中は大変お世話になりありがとうございました\par
    本年も変わらぬご厚誼のほどお願い申し上げます
    \begin{flushright}
        令和6年元旦
    \end{flushright}
}
\begin{itembox}[l]{\small Message}
    \vspace{3.1cm}
\end{itembox}
% \begin{framed}
%     \setlength{\baselineskip}{5pt}
%     {\fontsize{5px}{0px}\selectfont\noindent\input{jusyo.txt}}
%     \vspace{.3cm}
%     \begin{center}
%         {\LARGE 溝\ 口\ \  洸\ 熙}
%     \end{center}
% \end{framed}
\footnotetext[1]{ これらの等式は,ネットに転がっているもので,僕が考えたわけではありません.}
\noindent\scriptsize{\url{https://github.com/MIZOGUCHIKoki/LaTeX/tree/master/new_years_card/2025/main.tex}\\\hfill Compiled on Dec. 31, 2024 by Lua\LaTeX}

\newpage
\fontsize{10pt}{10pt}\selectfont
\begin{align*}
    2025 & = \sum_{k=1}^{9} k^3 = \left\{\frac{n(n+1)}{2}\right\}^2 \\
         & = 1^3 + 2^3 + 3^3 + 4^3                                  \\
         & \quad\quad + 5^3 + 6^3 + 7^3 + 8^3 + 9^3                 \\
\end{align*}
\scriptsize \(2025\)って,\(1\)から\(9\)までの立方数の和で表せるんですね\footnotemark[1].
\fontsize{10pt}{10pt}\selectfont
\begin{align*}
    2025 & = \sum_{i=1}^{9}\sum _{j=1}^{9}ij
\end{align*}
\scriptsize 2025って,九九の表の和で表せるんですね\footnotemark[1].
\lstinputlisting{sum_kuku.c}
\end{document}