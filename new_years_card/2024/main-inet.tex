\documentclass[twocolumn,fontsize=9pt]{jlreq}
\usepackage[papersize={148truemm, 100truemm}, top=15truemm,bottom=15truemm,left=10truemm,right=10truemm,headheight=20pt]{geometry}
\usepackage{fancyhdr,fancybox,ascmac}
\usepackage{xcolor,graphicx,url}
\usepackage{listings,framed}
\lstset{
    language=C,
    breaklines=true,
    breakindent=10pt,
    basicstyle=\ttfamily\tiny,
    frame=TB,
    keywordstyle={\color{red!70!black}\bfseries\ttfamily},
    commentstyle={\color{green!50!black}\ttfamily},
    stringstyle={\color{blue}\ttfamily},
    morekeywords={pipe, fork, pid_t, printf, sprintf, write, read},
}

\fancypagestyle{fancy}{
    \fancyhf{}
    \renewcommand{\headrulewidth}{0pt}
    \renewcommand{\footrulewidth}{0pt}
    \fancyfoot[L]{\scriptsize Updated: Dec. 31st, 2023}
    \fancyhead[R]{2024年}
    \fancyfoot[R]{\scriptsize Compiled by Lua\LaTeX}
}
\fancyput(-20truemm,-60truemm){
	\color{black!10}{
		\rotatebox{50}{
			\scalebox{10}{甲辰}     
		}
	}
}
\pagestyle{fancy}
\begin{document}
\twocolumn[
    \begin{center}
        {\LARGE 謹賀新年}
    \end{center}
    \vspace{0.5em}
]
{\small
    謹んで新年のご挨拶を申し上げます.\par
    旧年中は大変お世話になりありがとうございました.\par
    本年も変わらぬご厚誼のほどお願い申し上げます.
    \begin{flushright}
        令和6年元旦
    \end{flushright}
}
\begin{itembox}[l]{Message}
\end{itembox}
\begin{framed}
    \noindent{\small\url{https://github.com/MIZOGUCHIKoki/}}
    \vspace{.2cm}
    \begin{center}
        {\LARGE 溝\ 口\ \  洸\ 熙}
    \end{center}
\end{framed}
\noindent\tiny{\url{https://github.com/MIZOGUCHIKoki/LaTeX/blob/master/new_years_card/2024/main-inet.tex}}
\newpage
\lstinputlisting{eto.c}
\small{\texttt{\$ gcc eto.c -o eto \&\& ./eto \fbox{Enter}}\ \ \footnote{(無駄だけど)マルチプロセスで処理しています}\\
    \texttt{>> 2024年の干支は「甲辰」}}
\end{document}